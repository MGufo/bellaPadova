% Options for packages loaded elsewhere
\PassOptionsToPackage{unicode}{hyperref}
\PassOptionsToPackage{hyphens}{url}
%
\documentclass[
]{article}
\usepackage{lmodern}
\usepackage{amssymb,amsmath}
\usepackage{ifxetex,ifluatex}
\ifnum 0\ifxetex 1\fi\ifluatex 1\fi=0 % if pdftex
  \usepackage[T1]{fontenc}
  \usepackage[utf8]{inputenc}
  \usepackage{textcomp} % provide euro and other symbols
\else % if luatex or xetex
  \usepackage{unicode-math}
  \defaultfontfeatures{Scale=MatchLowercase}
  \defaultfontfeatures[\rmfamily]{Ligatures=TeX,Scale=1}
\fi
% Use upquote if available, for straight quotes in verbatim environments
\IfFileExists{upquote.sty}{\usepackage{upquote}}{}
\IfFileExists{microtype.sty}{% use microtype if available
  \usepackage[]{microtype}
  \UseMicrotypeSet[protrusion]{basicmath} % disable protrusion for tt fonts
}{}
\makeatletter
\@ifundefined{KOMAClassName}{% if non-KOMA class
  \IfFileExists{parskip.sty}{%
    \usepackage{parskip}
  }{% else
    \setlength{\parindent}{0pt}
    \setlength{\parskip}{6pt plus 2pt minus 1pt}}
}{% if KOMA class
  \KOMAoptions{parskip=half}}
\makeatother
\usepackage{xcolor}
\IfFileExists{xurl.sty}{\usepackage{xurl}}{} % add URL line breaks if available
\IfFileExists{bookmark.sty}{\usepackage{bookmark}}{\usepackage{hyperref}}
\hypersetup{
  pdftitle={Relazione progetto Programmazione a Oggetti A.A. 2019/20},
  pdfauthor={Gabriel Bizzo - 1170734, Marco Rosin - 1120673, Andrea Moscon - 1121217},
  hidelinks,
  pdfcreator={LaTeX via pandoc}}
\urlstyle{same} % disable monospaced font for URLs
\usepackage{graphicx}
\makeatletter
\def\maxwidth{\ifdim\Gin@nat@width>\linewidth\linewidth\else\Gin@nat@width\fi}
\def\maxheight{\ifdim\Gin@nat@height>\textheight\textheight\else\Gin@nat@height\fi}
\makeatother
% Scale images if necessary, so that they will not overflow the page
% margins by default, and it is still possible to overwrite the defaults
% using explicit options in \includegraphics[width, height, ...]{}
\setkeys{Gin}{width=\maxwidth,height=\maxheight,keepaspectratio}
% Set default figure placement to htbp
\makeatletter
\def\fps@figure{htbp}
\makeatother
\setlength{\emergencystretch}{3em} % prevent overfull lines
\providecommand{\tightlist}{%
  \setlength{\itemsep}{0pt}\setlength{\parskip}{0pt}}
\setcounter{secnumdepth}{-\maxdimen} % remove section numbering
\ifluatex
  \usepackage{selnolig}  % disable illegal ligatures
\fi

\title{Relazione progetto Programmazione a Oggetti A.A. 2019/20}
\author{Gabriel Bizzo - 1170734 \\ \and Marco Rosin - 1120673 \\ \and Andrea Moscon - 1121217}
\date{Relazione di Rosin Marco}

\begin{document}
\maketitle

\begin{center}
  \includegraphics[width=0.5\textwidth,height=\textheight]{./logo.png}
\end{center}
\begin{center}
  \includegraphics[width=0.75\textwidth,height=\textheight]{./uniPD_DM.jpg}
\end{center}

\hypertarget{abstract-e-funzionalituxe0}{%
\section{Abstract e Funzionalità}\label{abstract-e-funzionalituxe0}}

\hypertarget{descrizione-delle-gerarchie-utilizzate}{%
\section{Descrizione delle gerarchie
utilizzate}\label{descrizione-delle-gerarchie-utilizzate}}

\hypertarget{chiamate-polimorfe}{%
\section{Chiamate Polimorfe}\label{chiamate-polimorfe}}

\hypertarget{io}{%
\section{I/O}\label{io}}

\hypertarget{istruzioni-di-compilazione}{%
\section{Istruzioni di compilazione}\label{istruzioni-di-compilazione}}

\hypertarget{suddivisione-dei-compiti---ore-di-sviluppo}{%
\section{Suddivisione dei compiti - Ore di
sviluppo}\label{suddivisione-dei-compiti---ore-di-sviluppo}}

\hypertarget{ambiente-di-sviluppo}{%
\section{Ambiente di sviluppo}\label{ambiente-di-sviluppo}}

Il progetto è stato sviluppato in un'ambiente così configurato:

\begin{itemize}
\tightlist
\item
  \textbf{Sistema Operativo:} Microsoft Windows 10 Pro N 64bit (ver
  1903)
\item
  \textbf{Compilatore:} MinGW 5.3.0 32bit
\item
  \textbf{Qt} ver 5.9.5
\end{itemize}

Modello e container sono stati sviluppati utilizzando l'IDE Visual
Studio Code, poiché la maggior familiarità con esso ha permesso di
eliminare il tempo di adattamento a un nuovo IDE. Successivamente il
progetto è stato migrato a QtCreator in quanto la continua
configurazione manuale è stata ritenuta ingestibile.

La fase di testing/debugging è stata svolta simultaneamente in ambienti
Windows e Linux (quest'ultimo in particolare per la ricerca di
\emph{memory leak} tramite \emph{Valgrind}).

\end{document}
